%\documentclass{beamer}
\documentclass[handout]{beamer}
\usepackage{pnup}
\pgfpagesuselayout{3 on 1 with notes}[a4paper,border shrink=3mm]
\mode<presentation>
{
    %\usetheme{Darmstadt}
    \usetheme{Frankfurt}
    %\usetheme{Singapore}
    \usecolortheme{crane}
    \definecolor{craneblue}{RGB}{60,0,0}
%    \usecolortheme{seahorse}
    \setbeamercovered{transparent}
    \setbeamertemplate{items}[triangle]
    \setbeamertemplate{enumerate items}[default]

}


\usepackage[german]{babel}
\usepackage[utf8x]{inputenc}
\usepackage{booktabs}
\setcounter{tocdepth}{2}
\newlength{\tikey}
\newcommand\keystroke[1]{~\tikz[overlay]\node[inner sep=2pt, outer sep=2pt,anchor=text, rectangle, rounded corners=1mm,fill=black!20,draw] {#1};\phantom{#1}~}
\usepackage{outline}%
\def\labeloutlni{\theoutlni}%
\def\theoutlni{\protect{(\alph{outlni})}}%
\usepackage{booktabs}
\usepackage{graphicx}
\usepackage{enumerate}
\usepackage{wasysym}
\usepackage{txfonts}

\usepackage{tikz}
%\usebackgroundtemplate{%
%\begin{tikzpicture}
%  \node [rotate=30,scale=8.25,color=gray!20] at (current page.center) {DRAFT};
%\end{tikzpicture}}
\title{Taschenrechnerkurs für MAT183, Teil 2}

\author{Ueli Hartmann\\Mathias Weyland}

\date{Juni 2013}

\AtBeginSection[]
{
\subsection{}
  \begin{frame}<beamer>
    \begin{beamerboxesrounded}[shadow]{\inserttitle}
      \insertsection
    \end{beamerboxesrounded}
  \end{frame}
}

\setbeamertemplate{navigation symbols}{}
\begin{document}

\frame{\maketitle}


\section{Navigation}
\begin{frame}{Erste Schritte mit dem Stats List Editor}
\begin{itemize}
\item Starten mit \keystroke{APPS}
\item Navigieren mit den Pfeiltasten
\item Eingabe der Daten auf der gestrichelten Linie
\item Löschen einer Spalte durch Plazieren auf der Spaltenüberschrift, anschliessend \keystroke{CLEAR} und \keystroke{ENTER}
\item Bei Brüchen: \texttt{EXACT}-Mode ausschalten.
\end{itemize}
\end{frame}

\section{1/2-Var Stats}
\begin{frame}{Übersicht}
\begin{itemize}
\item Im Menü \keystroke{F4}, \texttt{1-Var Stats} oder \texttt{2-Var Stats}
\item Zum Berechnen von Messgrössen aus einer oder zwei Stichproben
\item Arbeiten mit Häufigkeiten möglich
\item Ausgabe: Stichprobenmittelwert, empirische Standardabweichung, Median etc.
\item Alternative Quartildefinition!
\item Vorsicht: \texttt{Sx $\ne\sigma$x}
\end{itemize}
\end{frame}

\begin{frame}{Aufgabe 3, Serie 2}
Die Messung der Körperlänge (auf 0.5cm genau) von 16 zehnjährigen Mädchen ergab
folgende Werte (in cm):

\begin{center}
131.5, 130, 137.5, 140.5, 132.5, 151, 138, 136, 133, 138, 134.5, 140, 132, 139, 133.5, 131
\end{center}
\begin{outline}
\item $[\dots]$
\item $[\dots]$
\item Lagemasse: Bestimmen Sie das arithmetische Mittel, den Median, sowie das 25\%-
und das 75\%-Quartil.
\item Streuungmasse: Bestimmen Sie die Variationsbreite und den Interdezilbereich. [\dots]
\item Streuungmasse: Bestimmen Sie die empirische Varianz und die Standardabweichung [\dots].
\end{outline}
\end{frame}

\section{Listen}
\begin{frame}{Übersicht}
\begin{itemize}
\item Formel in der Spaltenüberschrift eingeben, z.B. \texttt{list2=list1-mean(list1)}
\item Falls Bruch angegeben wird: \keystroke{$\Diamondblack$}~\keystroke{ENTER}
\item Gänsefüsschen (\texttt{"}) für automatische Korrektur
\end{itemize}
\end{frame}

%\section{Verteilungen}
%\begin{frame}{Übersicht}
%\begin{itemize}
%\item Im Menü \keystroke{F5}
%\item Dichtefunktion: PDF
%\item Wahrscheinlichkeitsverteilung: CDF
%\item Parameter angeben
%\item Die Grenzen der CDF angeben
%\item Freiheitsgrade angeben
%\item Für $t-$, $F-$, $\chi^2-$ und $\mathcal{N}-$Verteilung
%\item Inverse = Quantilfunktion
%\end{itemize}
%\end{frame}
%
%\begin{frame}{Wahrscheinlichkeitsverteilungen, p. 10f}
%\begin{beamerboxesrounded}[shadow]{Aufgabe 94, Serie 8}
%Die Länge (in mm) von Raupen einer bestimmten Art sei normalverteilt mit $\mu = 30$ und
%$\sigma = 5$. Es werden 240 Raupen untersucht. Wieviele davon erwarten Sie
%\begin{enumerate}[a)]
%\item mit einer Länge $\le$ 25 mm
%\item mit einer Länge zwischen 28 und 35 mm?
%\end{enumerate}
%\end{beamerboxesrounded}
%\end{frame}
%
%\begin{frame}{Inverse t-Verteilung, p. 10f}
%\begin{beamerboxesrounded}[shadow]{Aufgabe 109a), Serie 10}
%Die Zufallsgrösse T sei t-verteilt mit Freiheitsgrad $\nu = 15$. Bestimmen Sie
%mit Hilfe einer Tabelle die Zahl t so, dass $P[T \le t] = 0.95$ ist.
%\end{beamerboxesrounded}
%\end{frame}

\section{Regression (I)}
\begin{frame}{Übersicht}
\begin{itemize}
\item Im Menü \keystroke{F4} mit \texttt{Regressions$\RHD$}
\item Zwei Modelle: \texttt{LinReg(a+bx)} oder \texttt{LinReg(ax+b)}
\item Schätzung der Parameter und des Bestimmtheitsmasses $R^2$
\item Test ob wahre Steigung $= 0$: \keystroke{F6} und \texttt{A:LinRegTTest}
\end{itemize}
\end{frame}

\begin{frame}{Theorie}
\begin{enumerate}
\item Stichprobe: $(x_1,Y_1)\ldots(x_n, Y_n)$
\item Berechne:
$$
\bar{Y},\; \bar{x},\; \sum_{i=1}^n (Y_i-\bar{Y})(x_i-\bar{x}),\; \sum_{i=1}^n(x_i-\bar{x})^2
$$

\item Berechne:
$$
\hat{\beta}_1 = \frac{\sum_{i=1}^n
(Y_i-\bar{Y})(x_i-\bar{x})}{\sum_{i=1}^n(x_i-\bar{x})^2}=
\frac{\sum_{i=1}^n \left(x_iY_i\right)-n\bar{x}\bar{Y}}{\sum_{i=1}^n \left(x_i^2\right)-n\bar{x}^2},\;
\hat{\beta_0} = \bar{Y}-\hat{\beta}_1\bar{x}
$$

\item Geradengleichung $\hat{Y}=\hat{\beta}_0+\hat{\beta}_1 x_i$

\item Korrelationskoeffizient $r = \frac{\sum_{i=1}^n
    (Y_i-\bar{Y})(x_i-\bar{x})}{\sqrt{\sum_{i=1}^n(Y_i-\bar{Y})^2\sum_{i=1}^n(x_i-\bar{x})^2}}$
\end{enumerate}
\end{frame}

\begin{frame}{Einfache Lineare Regression, Parameterschätzung}
\begin{beamerboxesrounded}[shadow]{Aufgabe 5, Serie 12}
Wir betrachten die folgenden Messwerte zu fünf der Galápagos-Inseln aus der
vorherigen Aufgabe:

\begin{center}{\scriptsize\begin{tabular}{l|ccccc}\toprule
Insel & Baltra & Eden & Isabela & Pinta & Santa Fé\\\midrule
$x_i$&
 1.12 & 0.77 & 0.82 & 1.56 & 1.22\\
$y_i$&
  58 & 44 & 56 & 45 & 62\\
\bottomrule
\end{tabular}}\end{center}

\begin{outline}
\item $[\dots]$ Bestimmen Sie die Gleichung der Regressionsgeraden
$y=\hat{\beta_0}+\hat{\beta_1}x$ [\dots].
\textcolor{black!20}{\item Hat die Fläche einen Einfluss auf die Anzahl der Pflanzenarten?
Testen Sie $\mathcal{H}_0: \beta_1 = 0$ gegen $\mathcal{H}_1: \beta_1 \ne 0$
zum Signifikanzniveau 5\%.}

\item Berechnen Sie den Korrelationskoeffizienten zwischen den Daten
$(x_i)_{i=1}^5$ und $(y_i)_{i=1}^5$.
\end{outline}
\end{beamerboxesrounded}
\end{frame}

\section{T-Test}
\begin{frame}{Übersicht}
\begin{itemize}
\item Im Menü \keystroke{F6}
\item One Sample- und Two Sample T-Test.
\item Mit Schätzern (\texttt{Stats}) oder Daten (\texttt{Data})
\item Eingabe der Prüfgrösse $\mu_0$ und der \textbf{Alternativ}-Hypothese
\item Ausgabe von t- und p-Wert sowie $\bar{x}$, $n$ und empirische Standardabweichung
\end{itemize}
\end{frame}

\begin{frame}{One Sample T-Test, Theorie}
\begin{enumerate}
\item Stichprobe: $x_1\ldots x_n$, Prüfgrösse: $\mu_0$.
\item Berechne:
$$
\bar{x}=\frac{1}{n}\sum_{i=1}^n x_i,\;
s=\sqrt{\frac{1}{n-1}\sum_{x=1}^{n}(x_i-\bar{x})^2},\;
\nu=n-1
$$
\item Berechne:
$$
t_\text{obs}=\frac{\bar{x}-\mu_0}{\frac{s}{\sqrt{n}}} = 
\frac{\sqrt{n}\left(\bar{x}-\mu_0\right)}{s}
$$
\item Inferenz mit Tabelle oder p-Wert. ($t_\text{obs}\sim T_\nu$).
\end{enumerate}
\end{frame}

\begin{frame}{One Sample T-Test mit \texttt{Data}}
\begin{beamerboxesrounded}[shadow]{Aufgabe 2, Serie 11}
Eine Abfüllmaschine für Puderzucker war so eingestellt, dass das mittlere Abfüllgewicht
pro Packung 150 g betrug. Eine Stichprobe von 8 Packungen lieferte folgende Werte (in
g)

\begin{center}
130, 110, 100, 150, 160, 130, 110, 150
\end{center}

\begin{outline}
\item Testen Sie mit einem zweiseitigen Test die Hypothese, dass das
Durchschnittsgewicht Pro Packung 150 g beträgt.
\item Es  wird vermutet, die Packungen wögen im Mittel weniger als 150 g. Testen Sie
mit einem einseitigen Test auch diese Hypothese. Wie sind $\mathcal{H}_0$ und 
$\mathcal{H}_1$ zu wählen?
\end{outline}
Arbeiten Sie sowohl bei (a) als auch bei (b) mit $\alpha = 5\%$.
\end{beamerboxesrounded}
\end{frame}

%\begin{frame}{Paired T-Test, Theorie}
%\begin{enumerate}
%\item Stichprobe: $(a_1, b_1)\ldots (a_n,b_n)$, Prüfgrösse: $\mu_0$.
%\item Berechne $x_i=a_i-b_i$
%\item Führe One Sample  T-Test mit $x_i$ durch.
%\end{enumerate}
%\vfill
%\textit{Hinweis auf Serie 11:} Ein t-Test für zwei gepaarte Stichproben ist ein
%t-Test für die Differenzen der gepaarten Daten, also eigentlich ein t-Test für
%eine Stichprobe.
%\end{frame}
%
%\begin{frame}{Paired T-Test, Beispiel}
%\begin{beamerboxesrounded}[shadow]{Aufgabe 123, Serie 11}
%Zehn Personen führten mit folgendem Ergebnis uber den gleichen Zeitraum eine
%Diät durch:
%
%\vspace{2mm}\begin{center}{\tiny\begin{tabular}{l|cccccccccc}
%Person Nr.&1&2&3&4&5&6&7&8&9&10\\\hline
%Gewicht vorher&
%103.7&95.2&87.6&97.7&76.8&110.9&85.1&95.3&101.3&99.8\\
%Gewicht nachher&
%99.8&92.3&88.2&96.5&74.0&106.4&83.4&95.0&100.0&95.3\\
%\end{tabular}}\end{center}\vspace{2mm}
%
%Die Erfinderin der Diät behauptet, dass diese tatsächlich eine Gewichtsabnahme
%bewirke.  Prüfen Sie diese Behauptung mit einem statistischen Test nach, 
%\begin{outline}
%\item mit dem Signifikanzniveau 0.05, 
%\item mit dem Signifikanzniveau 0.0005. 
%\end{outline}
%
%Geben Sie dabei die Null- und die Alternativhypothese explizit an.
%\end{beamerboxesrounded}
%\end{frame}

\begin{frame}{Two Sample T-Test, Theorie}
\begin{enumerate}
\item Stichproben: $x_1\ldots x_{n1}$ und $y_1\ldots y_{n2}$.
\item Berechne: $\bar{x}$, $\bar{y}$, $s_1$ und $s_2$ wie üblich.
\item Berechne: 
$$
\nu=n_1+n_2-2,\;
s=\sqrt{\frac{(n_1-1)s_1^2+(n_2-1)s_2^2}{\nu}}
$$

\item Berechne:
$$
t_\text{obs}=\frac{\bar{x}-\bar{y}}{s\sqrt{1/n_1+1/n_2}}
$$
\item Inferenz mit Tabelle oder p-Wert. ($t_\text{obs}\sim T_\nu$).
\end{enumerate}
\end{frame}

\begin{frame}{Two Sample T-Test, Beispiel}
\begin{beamerboxesrounded}[shadow]{Aufgabe 4, Serie 11}
In einer Studie wird der Einfluss eines Düngemittels auf eine bestimmte Pflanzenart 
untersucht. Dazu wurde die erste Gruppe mit 12 Töpfen mit je einem Keimling dieser 
Pflanzenart regelmässig mit dem Düngemittel versorgt, und 8 Töpfe mit je einem Keimling
ohne Dünger gegossen. Die Länge in mm nach einem Monat waren bei der ersten Gruppe
\begin{center}
245, 240, 236, 243, 247, 238, 239, 248, 238, 240, 244, 237
\end{center}
und bei der zweiten Gruppe
\begin{center}
235, 233, 236, 234, 231, 242, 237, 230 .
\end{center}

\begin{outline}
\item Prüfen Sie die Hypothese $\mathcal{H}_0$, dass die beiden Grundgesamtheiten gleiche
Erwartungswerte haben, mit $\alpha = 5\%$.
\item [\dots]
\end{outline}
\end{beamerboxesrounded}
\end{frame}

%\section[Test bei E-W'keiten]{Tests bei Erfolgswahrscheinlichkeiten}
%\begin{frame}{Übersicht}
%\begin{itemize}
%\item Im Menü \keystroke{F6}
%\item \texttt{5:1-PropZTest\ldots} für den Einstichproben-Fall.
%\item \texttt{6:2-PropZTest\ldots} nicht benutzen!
%\item In der Regel \texttt{Data}.
%\end{itemize}
%\end{frame}
%
%\begin{frame}{Test bei Erfolgswahrscheinlichkeiten, Theorie}
%Für eine Stichprobe:
%\begin{enumerate}
%\item Gegeben sind $k$, $n$ und $p_0$.
%\item Berechne: 
%$$
%z_\text{obs}=\left| \frac{\frac{k}{n}-p_0}{\sqrt{\frac{p_0(1-p_0)}{n}}}\right|
%$$
%\item Inferenz mit Tabelle oder p-Wert ($t_\text{obs}\sim\mathcal{N}(0,1)$).
%\end{enumerate}
%\end{frame}
%
%\begin{frame}{Test bei Erfolgswahrscheinlichkeiten, Beispiel}
%\begin{beamerboxesrounded}[shadow]{Aufgabe 66(a), Serie 11}
%Bei einer Umfrage unter der weissen Bevölkerung der USA wurde ermittelt, ob
%die Leute mit der Amtsführung von George W. Bush zufrieden waren oder nicht.
%Es wurden 2000 Leute befragt. Davon gaben 890 Personen an, sie seien mit der
%Amtsführung von George W. Bush zufrieden. Die restlichen 1110 Personen gaben an,
%sie seien unzufrieden. Testen Sie mit einem Signifikanzniveau von 1\%, ob aufgrund
%dieser Umfrage die Meinung haltbar ist, dass die Hälfte der weissen Bevölkerung der
%USA mit der Amtsführung des Präsidenten zufrieden war. [\dots]
%\end{beamerboxesrounded}
%\end{frame}

\section[KI]{Konfidenzintervalle}
\begin{frame}{Übersicht}
\begin{itemize}
\item Im Menü \keystroke{F7}
\item \texttt{2:TInterval\ldots} für den Einstichproben-Fall und
\item \texttt{4:2SampleTInt\ldots} für den Zweistichproben-Fall.
\item Analog zu T-Tests
\item \texttt{Stats} oder \texttt{Data}
\end{itemize}
\end{frame}


\begin{frame}{KI für einen Erwartungswert mit \texttt{Stats}}
\begin{beamerboxesrounded}[shadow]{Aufgabe 6, Serie 10}
Die Bestimmung des Durchmessers der Blüten von Heckenrosen 
(\textit{Rosa canina}) ergab einen Mittelwert von 37.5 mm mit 
einer empirischen Varianz von 156.25 mm$^2$. Geben Sie
das 95\%-Konfidenzintervall für den mittleren Blütendurchmesser an 
für den Fall, wo die Messungen an 
\begin{outline}
\item 20 
\item 2000 Blüten durchgeführt wurden.
\end{outline}
\end{beamerboxesrounded}
\end{frame}

\begin{frame}{KI für einen Erwartungswert mit \texttt{Data}}
\begin{beamerboxesrounded}[shadow]{Aufgabe 2, Serie 10}

An einer Stichprobe von 7 Schmetterlingen einer bestimmten Art wurden folgende 
Flügelspannweiten (in mm) gemessen:

\begin{center}
34, 28, 36, 24, 32, 32, 38
\end{center}

Nehmen Sie Normalverteilung an.

\begin{outline}
\item Schätzen Sie Erwartungswert, Varianz, Standardabweichung und Standardfehler der Grundgesamtheit.
\item Bestimmen Sie ein 95\%-Konfidenzintervall für die mittlere Flügelspannweite.
\end{outline}
\end{beamerboxesrounded}
\end{frame}

\begin{frame}{KI für Differenz zweier Erwartungswerte mit \texttt{Stats}}
\begin{beamerboxesrounded}[shadow]{Aufgabe 5, Serie 10}
In einer Studie wird der Einfluss der UV-Strahlung auf eine bestimmte Pflanzenart 
untersucht. Dazu werden 31 Töpfe mit je einem Keimling dieser Pflanzenart regelmässig
mit UV-Lampen bestrahlt, und 31 Töpfe mit je einem Keimling wachsen unter normalen
Bedingungen.

Nach einem Monat wird die Länge jeder Pflanze gemessen. Wir stellen fest, dass die unbestrahlten 
Pflanzen im Durchschnitt 30 cm gross sind, mit einer empirischen Standardabweichung von 4 cm, 
und dass die bestrahlten Pflanzen eine durchschnittliche Grösse von 24 cm erreicht haben, mit 
einer Standardabweichung von 3 cm.

Wie gross ist das 95\%-Konfidenzintervall für die Differenz der Mittelwerte?
\end{beamerboxesrounded}
\end{frame}

%\begin{frame}{KI für Differenz zweier Wahrscheinlichkeiten mit \texttt{Stats}}
%\begin{beamerboxesrounded}[shadow]{Aufgabe 4, Serie 10}
%\small{In einer Studie wird ein neues alternatives Medikament gegen eine bestimmte Krankheit
%untersucht. Es nehmen 500 Personen, die an einer Krankheit leiden, an der Studie teil.
%Von diesen Personen werden 300 zufällig ausgewählt. Sie bekommen das neue Medikament
%verabreicht. Die verbleibenden 200 Personen bilden die Kontrollgruppe, d.h. sie nehmen
%weiterhin ein herkömmliches Medikament ein. Nach einem Jahr wird kontrolliert, wie sich
%der Gesundheitszustand der Teilnehmer verändert hat. In der ersten Gruppe stellen wir
%fest, dass sich der Gesundheitszustand bei 201 Personen verbessert hat und bei den 
%verbleibenden 99 gleich geblieben ist oder sich gar verschlechtert hat. Bei der Kontrollgruppe
%hat sich der Gesundheitszustand bei 121 Personen verbessert, und bei den restlichen 79
%Leuten ist er gleich geblieben oder hat sich verschlechtert. Mit $p_1$ bezeichnen wir die
%Wahrscheinlichkeit der Verbesserung des Gesundheitszustandes (innerhalb eines Jahres)
%einer Person der ersten Gruppe und mit $p_2$ die entsprechende Wahrscheinlichkeit für
%eine Person in der zweiten Gruppe. Bestimmen Sie das 95\%-Konfidenzintervall der Differenz.}
%\end{beamerboxesrounded}
%\end{frame}
%
\section{ANOVA}
\begin{frame}{Übersicht}
\begin{itemize}
\item Im Menü \keystroke{F6}: \texttt{C:ANOVA}
\item Eingabe: Eine Spalte pro Gruppe
\item Ausgabe: ANOVA-Tabelle sowie $F-$ und $p-$Wert
\end{itemize}
\end{frame}

\begin{frame}{Theorie}
\begin{outline}
\item Gegeben sind $k$ Gruppen, in der $i$-ten Gruppe $n_i$, Beobachtungen
$y_{i\,1}, y_{i\,2}, \ldots, y_{i\,ni}$.
\item Berechne:
$$
n=\sum_{i=1}^k n_i,\;
\bar{Y}_{i.}=\frac{1}{n_i}\sum_{j=1}^{n_i}Y_{ij},\; 
\bar{Y}_{..}=\frac{1}{n}\sum_{i=1}^k\sum_{j=1}^{n_i}Y_{ij}
$$
\item Berechne:
$$
MS_G=\frac{1}{k-1}\underbrace{\sum_{i=1}^k n_i(\bar{Y}_{i.}-\bar{Y}_{..})^2}_{SS_G},\;
MS_E=\frac{1}{n-k}\underbrace{\sum_{i=1}^k\sum_{j=1}^{n_i}(Y_{ij}-\bar{Y}_{i.})^2}_{SS_E}
$$
\item $V=\frac{MS_G}{MS_E}$
\item Inferenz mit Tabelle oder p-Wert. ($V\sim F_{\underbrace{k-1}_{df_G},\underbrace{n-k}_{df_E}}$).
\end{outline}
\end{frame}


\begin{frame}{Einfache Varianzanalyse}
\begin{beamerboxesrounded}[shadow]{Aufgabe 2, Serie 12}
Wir messen das Gewicht von Regenbogenforellen von drei verschiedenen Seen. Pro See
vermessen wir vier Fische und notieren die erhaltenen Werte (in g)

\vspace{2mm}\begin{center}{\scriptsize\begin{tabular}{ccc}\toprule
See 1 & See 2 & See 3\\\midrule
132 & 172 & 153\\
169 & 163 & 162\\
158 & 161 & 151\\
164 & 149 & 158\\
\bottomrule
\end{tabular}}\end{center}\vspace{2mm}

Unterscheiden sich die Fische hinsichtlich ihres Gewichtes je nachdem in welchem
See sie leben? Führen Sie zur Beantwortung dieser Frage eine Varianzanalyse durch.
Wählen Sie als Signifikanzniveau 5\%.
\end{beamerboxesrounded}
\end{frame}

\section{Regression (II)}

\begin{frame}{Theorie (Inferenz)}
\begin{enumerate}
\item Berechne:
$$
\hat{\sigma}^2 = \frac{1}{n-2}\sum_{i=1}^n (Y_i-\hat{Y}_i)^2
$$
\item Berechne für $\mathcal{H}_0: \beta_1=b$:
$$
t_\text{obs}=\frac{\hat\beta_1-b}{\sqrt{\hat\sigma^2/\sum_{i=1}^n(x_i-\bar{x})^2}}
$$
\item Inferenz mit Tabelle oder p-Wert. ($t_\text{obs}\sim T_{n-2}$).
\end{enumerate}
\end{frame}

\begin{frame}{Einfache Lineare Regression, Inferenz}
\begin{beamerboxesrounded}[shadow]{Aufgabe 5, Serie 12}
Wir betrachten die folgenden Messwerte zu fünf der Galápagos-Inseln aus der
vorherigen Aufgabe:

\begin{center}{\scriptsize\begin{tabular}{l|ccccc}\toprule
Insel & Baltra & Eden & Isabela & Pinta & Santa Fé\\\midrule
$x_i$&
 1.12 & 0.77 & 0.82 & 1.56 & 1.22\\
$y_i$&
  58 & 44 & 56 & 45 & 62\\
\bottomrule
\end{tabular}}\end{center}

\begin{outline}
\item $[\dots]$ Bestimmen Sie die Gleichung der Regressionsgeraden
$y=\hat{\beta_0}+\hat{\beta_1}x$ [\dots].
\item \textbf{Hat die Fläche einen Einfluss auf die Anzahl der Pflanzenarten?
Testen Sie $\mathcal{H}_0: \beta_1 = 0$ gegen $\mathcal{H}_1: \beta_1 \ne 0$
zum Signifikanzniveau 5\%.}
\item Berechnen Sie den Korrelationskoeffizienten zwischen den Daten
$(x_i)_{i=1}^5$ und $(y_i)_{i=1}^5$.
\end{outline}
\end{beamerboxesrounded}
\end{frame}

%\section{$\chi^2$-Test}
%\begin{frame}{$\chi^2$-Test auf gegebene Verteilung}
%\begin{beamerboxesrounded}[shadow]{Aufgabe 127, Serie 12}
%In einem grossen Wald mit gleichartigem Gelände und Baumbestand wird das
%Vorkommen der Heidelbeere untersucht. Auf quadratischen Flächen von je 1 m$^2$
%Grösse erhält man durch Auszählung der Pflanzen:
%
%\vspace{2mm}\begin{center}{\scriptsize\begin{tabular}{l|cccccccc} 
%Anzahl der Planzen pro m$^2$&0&1&2&3&4&5&6&$\ge 7$\\\hline
%Anzahl der Flächenstücke&15&18&11&4&1&0&1&0
%\end{tabular}}\end{center}\vspace{2mm}
%
%Darf eine Poissonverteilung mit Erwartungswert 1.2 angenommen werden?
%\end{beamerboxesrounded}
%\end{frame}

%\begin{frame}{Übersicht (Test auf Unabhängigkeit)}
%\begin{itemize}
%\item Im Menü \keystroke{F6} mit \texttt{8:Chi2 2-way}
%\item Kontingenztafel als Matrix eingeben:
%\begin{itemize}
%\item Zeileneinträge mit Kommata (\texttt{,}) trennen.
%\item Zeilen mit eckigen Klammern umfassen.
%\item Alles durch ein weiteres Paar eckige Klammern umfassen.
%\end{itemize}
%Beispiel:
%
%\begin{center}
%$\begin{bmatrix}1 & 2\\ 3 & 4\\\end{bmatrix}\Leftrightarrow$\texttt{[[1,2][3,4]]} 
%\end{center}
%
%\item Matrix der erwarteten Werte kann abgespeichert werden.
%\end{itemize}
%\end{frame}
%
%\begin{frame}{Übersicht (Test auf Unabhängigkeit)}
%\begin{enumerate}
%\item Kontingenztafel der Häufigkeiten $N_{ij}$ aufstellen $\Rightarrow$ $m$ Zeilen und $r$ Spalten.
%\item Summe $N$ aller Häufigkeiten berechnen.
%\item Randhäufigkeiten $N_{i.}$ und $N_{j.}$ berechnen.
%\item Erwartete Werte mit
%$$
%E_{ij}=\frac{n_{.i}n_{j.}}{n}
%$$
%berechnen.
%\item
%$$
%u=\sum_{i,j}\frac{(N_{ij}-E_{ij})^2}{E_{ij}}
%$$
%\item Inferenz mit Tabelle oder p-Wert. ($u\sim \chi^2_{(m-1)(r-1)}$).
%\end{enumerate}
%\end{frame}
%
%\begin{frame}{$\chi^2$-Test auf Unabhängigkeit}
%\begin{beamerboxesrounded}[shadow]{Aufgabe 127, Serie 12}
%Am 27. Januar 1987 berichtete die New York Times auf der Titelseite von den Resultaten
%einer Studie uber die präventive Wirkung von Aspirin gegen Herzinfarkte bei Männern
%mittleren Alters. Für die Studie wurden 22071 Männer mittleren Alters zufällig je einer
%von zwei Gruppen zugeordnet. Der einen Gruppe wurde Aspirin verabreicht, der anderen
%ein Placebo. Von 11037 Personen, die Aspirin eingenommen hatten, bekamen 104 einen
%Herzinfarkt; von den 11034 Personen, welchen ein Placebo verabreicht wurde, erlitten 189
%einen Herzinfarkt. Besteht ein signifikanter Unterschied ($\alpha = 0.001$)?
%\end{beamerboxesrounded}
%\end{frame}
\end{document}
