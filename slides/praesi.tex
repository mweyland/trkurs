%\documentclass{beamer}
\documentclass[handout]{beamer}
\usepackage{pnup}
\pgfpagesuselayout{3 on 1 with notes}[a4paper,border shrink=3mm]
\mode<presentation>
{
    %\usetheme{Darmstadt}
    \usetheme{Frankfurt}
    %\usetheme{Singapore}
    \usecolortheme{crane}
    \definecolor{craneblue}{RGB}{60,0,0}
%    \usecolortheme{seahorse}
    \setbeamercovered{transparent}
    \setbeamertemplate{items}[triangle]
    \setbeamertemplate{enumerate items}[default]

}


\usepackage[german]{babel}
\usepackage[utf8x]{inputenc}
\usepackage{booktabs}
\setcounter{tocdepth}{2}
\newlength{\tikey}
\newcommand{\keystroke}[1]{\settowidth{\tikey}{\scriptsize #1}\psframebox[framearc=0.2]{\parbox{\tikey}{\scriptsize #1}}}

\usepackage{outline}%
\def\labeloutlni{\theoutlni}%
\def\theoutlni{\protect{\alph{outlni})}}%
\usepackage{booktabs}
\usepackage{graphicx}
\usepackage{enumerate}
\usepackage{pstricks}
\usepackage{txfonts}
\usepackage{wasysym}

%\usepackage{tikz}
%\usebackgroundtemplate{%
%\begin{tikzpicture}
%  \node [rotate=30,scale=8.25,color=gray!20] at (current page.center) {DRAFT};
%\end{tikzpicture}}
\title{Taschenrechnerkurs für MAT183, Teil 2}

\author{Mathias Weyland}

\date{7. -- 15. Juni 2011}

\AtBeginSection[]
{
\subsection{}
  \begin{frame}<beamer>
    \begin{beamerboxesrounded}[shadow]{\inserttitle}
      \insertsection
    \end{beamerboxesrounded}
  \end{frame}
}

\setbeamertemplate{navigation symbols}{}
\begin{document}

\frame{\maketitle}


\section{Navigation}
\begin{frame}{Erste Schritte mit dem Stats List Editor}
\begin{itemize}
\item Starten mit \keystroke{APPS}
\item Navigieren mit den Pfeiltasten
\item Eingabe der Daten auf der gestrichelten Linie
\item Löschen einer Spalte durch Plazieren auf der Spaltenüberschrift, anschliessend \keystroke{CLEAR} und \keystroke{ENTER}
\item Bei Brüchen: \texttt{EXACT}-Mode ausschalten.
\end{itemize}
\end{frame}

\section{1/2-Var Stats}
\begin{frame}{Übersicht}
\begin{itemize}
\item Im Menü \keystroke{F4}, \texttt{1-Var Stats} oder \texttt{2-Var Stats}
\item Zum Berechnen von Messgrössen aus einer oder zwei Stichproben
\item Arbeiten mit Häufigkeiten möglich
\item Ausgabe: Stichprobenmittelwert, empirische Standardabweichung, Median etc.
\item Alternative Quartildefinition!
\item Vorsicht: \texttt{Sx $\ne\sigma$x}
\end{itemize}
\end{frame}

\begin{frame}{Statistische Messgrössen}
\begin{beamerboxesrounded}[shadow]{Aufgabe 105, Serie 10}

An einer Stichprobe von 12 Spatzen einer bestimmten Art wurden folgende
Flügelspannweiten (in mm) gemessen:

$$
245,\; 240,\; 236,\; 243,\; 247,\; 238,\; 239,\; 248,\; 238,\; 240,\; 244,\; 237
$$

\begin{outline}
\item Schätzen Sie Erwartungswert, Varianz und Standardfehler für die Grundgesamtheit.
\item \dots
\end{outline}
\end{beamerboxesrounded}
\end{frame}

%\begin{frame}[fragile]{Messgrössen mit Häufigkeiten}
%\begin{beamerboxesrounded}[shadow]{Aufgabe 104, Serie 10}
%Eine Untersuchung ergab folgende Daten:
%\begin{center}\begin{tabular}{l|rrrrrrrr}\toprule
%Messwert:          &15& 17& 19& 25& 30& 14& 29& 38\\
%Absolute Häufigkeit& 2&  6&  4& 10&  7&  2&  1&  3\\\bottomrule
%\end{tabular}\end{center}
%
%Schätzen Sie Erwartungswert, Varianz und Standardabweichung der zugehörigen
%Grundgesamtheit.
%\end{beamerboxesrounded}
%\end{frame}

\section{Listen}
\begin{frame}{Übersicht}
\begin{itemize}
\item Formel in der Spaltenüberschrift eingeben, z.B. \texttt{list2=list1-mean(list1)}
\item Falls Bruch angegeben wird: \keystroke{$\Diamondblack$}~\keystroke{ENTER}
\item Gänsefüsschen (\texttt{"}) für automatische Korrektur
\end{itemize}
\end{frame}

%\begin{frame}{Relative Häufigkeiten}
%\begin{beamerboxesrounded}[shadow]{Eigene Aufgabe}
%Berechnen Sie die relativen Häufigkeiten aus den absoluten Häufigkeiten der letzten Aufgabe.
%\end{beamerboxesrounded}
%\end{frame}

%\section{Verteilungen}
%\begin{frame}{Übersicht}
%\begin{itemize}
%\item Im Menü \keystroke{F5}
%\item Dichtefunktion: PDF
%\item Wahrscheinlichkeitsverteilung: CDF
%\item Parameter angeben
%\item Die Grenzen der CDF angeben
%\item Freiheitsgrade angeben
%\item Für $t-$, $F-$, $\chi^2-$ und $\mathcal{N}-$Verteilung
%\item Inverse = Quantilfunktion
%\end{itemize}
%\end{frame}
%
%\begin{frame}{Wahrscheinlichkeitsverteilungen, p. 10f}
%\begin{beamerboxesrounded}[shadow]{Aufgabe 94, Serie 8}
%Die Länge (in mm) von Raupen einer bestimmten Art sei normalverteilt mit $\mu = 30$ und
%$\sigma = 5$. Es werden 240 Raupen untersucht. Wieviele davon erwarten Sie
%\begin{enumerate}[a)]
%\item mit einer Länge $\le$ 25 mm
%\item mit einer Länge zwischen 28 und 35 mm?
%\end{enumerate}
%\end{beamerboxesrounded}
%\end{frame}
%
%\begin{frame}{Inverse t-Verteilung, p. 10f}
%\begin{beamerboxesrounded}[shadow]{Aufgabe 109a), Serie 10}
%Die Zufallsgrösse T sei t-verteilt mit Freiheitsgrad $\nu = 15$. Bestimmen Sie
%mit Hilfe einer Tabelle die Zahl t so, dass $P[T \le t] = 0.95$ ist.
%\end{beamerboxesrounded}
%\end{frame}

\section{Regression (I)}
\begin{frame}{Übersicht}
\begin{itemize}
\item Im Menü \keystroke{F4} mit \texttt{Regressions$\RHD$}
\item Zwei Modelle: \texttt{LinReg(a+bx)} oder \texttt{LinReg(ax+b)}
\item Schätzung der Parameter und des Bestimmtheitsmasses $R^2$
\item Test ob wahre Steigung $= 0$: \keystroke{F6} und \texttt{A:LinRegTTest}
\end{itemize}
\end{frame}

\begin{frame}{Theorie}
\begin{enumerate}
\item Stichprobe: $(x_1,Y_1)\ldots(x_n, Y_n)$
\item Berechne:
$$
\bar{Y},\; \bar{x},\; \sum_{i=1}^n (Y_i-\bar{Y})(x_i-\bar{x}),\; \sum_{i=1}^n(x_i-\bar{x})^2
$$

\item Berechne:
$$
\hat{\beta} = \frac{\sum_{i=1}^n (Y_i-\bar{Y})(x_i-\bar{x})}{\sum_{i=1}^n(x_i-\bar{x})^2},\;
\hat{\alpha} = \bar{Y}-\hat{\beta}\bar{x}
$$

\item Geradengleichung $\hat{Y}=\hat{\alpha}+\hat{\beta}x_i$
\end{enumerate}
\end{frame}

\begin{frame}{Einfache Lineare Regression}
\begin{beamerboxesrounded}[shadow]{Aufgabe 134, Serie 12}
Wir betrachten die folgenden Messwerte zu fünf der Galápagos-Inseln aus Aufgabe
133:

\vspace{2mm}\begin{center}{\scriptsize\begin{tabular}{l|ccccc}\toprule
Insel & Bartolomé & Darwin & Seymour & Tortuga & Wolf\\\midrule
$x_i$&
1.24&2.33&1.84&1.24&2.85\\
$y_i$&
31&10&44&16&21\\
\bottomrule
\end{tabular}}\end{center}\vspace{2mm}

\begin{itemize}
\item[b)] Bestimmen Sie die Gleichung der Regressionsgeraden
$y=\hat{\alpha}+\hat{\beta}x$ durch diese Punktpaare [\dots].
\item[d)] Hat die Fläche Einfluss auf die Zielgrösse $Y$?
\end{itemize}
Testen Sie H0 : $\beta = 0$ gegen H1 : $\beta \ne 0$
zum Signifikanzniveau 5\%.
\end{beamerboxesrounded}
\end{frame}

\section{T-Test}
\begin{frame}{Übersicht}
\begin{itemize}
\item Im Menü \keystroke{F6}
\item One Sample-, Two Sample und Paired T-Test
\item Mit Schätzern (\texttt{Stats}) oder Daten (\texttt{Data})
\item Eingabe der Prüfgrösse $\mu_0$ und der \textbf{Alternativ}-Hypothese
\item Ausgabe von t- und p-Wert sowie $\bar{x}$, $n$ und empirische Standardabweichung
\end{itemize}
\end{frame}

\begin{frame}{One Sample T-Test, Theorie}
\begin{enumerate}
\item Stichprobe: $x_1\ldots x_n$, Prüfgrösse: $\mu_0$.
\item Berechne:
$$
\bar{x}=\frac{1}{n}\sum_{i=1}^n x_i,\;
s=\sqrt{\frac{1}{n-1}\sum_{x=1}^{n}(x_i-\bar{x})^2},\;
\nu=n-1
$$
\item Berechne:
$$
t_\text{obs}=\frac{\bar{x}-\mu_0}{s/\sqrt{n}}
$$
\item Inferenz mit Tabelle oder p-Wert. ($t_\text{obs}\sim T_\nu$).
\end{enumerate}
\end{frame}

\begin{frame}{One Sample T-Test, Beispiel}
\begin{beamerboxesrounded}[shadow]{Aufgabe 119, Serie 11}
Acht mit Mineralwasser gefüllte Halbliterflaschen der gleichen Herstellerfirma
wurden gewogen. Man erhielt folgende Gewichte in Kilogramm

\begin{center}
0.49, 0.46, 0.45, 0.49, 0.47, 0.50, 0.51, 0.45
\end{center}
\begin{outline}
\item Testen Sie die Hypothese, dass das Durchschnittsgewicht der Flaschen 0.50
kg beträgt (zweiseitiger Test).
\item Es wird vermutet, die Flaschen wögen im Mittel weniger als 0.50 kg.
Testen Sie auch diese Hypothese (einseitiger Test, wie sind H0 und H1 zu wählen?).
\end{outline}
Wählen Sie $\alpha$ = 5\%.
\end{beamerboxesrounded}
\end{frame}

\begin{frame}{Paired T-Test, Theorie}
\begin{enumerate}
\item Stichprobe: $(a_1, b_1)\ldots (a_n,b_n)$, Prüfgrösse: $\mu_0$.
\item Berechne $x_i=a_i-b_i$
\item Führe One Sample  T-Test mit $x_i$ durch.
\end{enumerate}
\vfill
\textit{Hinweis auf Serie 11:} Ein t-Test für zwei gepaarte Stichproben ist ein
t-Test für die Differenzen der gepaarten Daten, also eigentlich ein t-Test für
eine Stichprobe.
\end{frame}

\begin{frame}{Paired T-Test, Beispiel}
\begin{beamerboxesrounded}[shadow]{Aufgabe 123, Serie 11}
Zehn Personen führten mit folgendem Ergebnis uber den gleichen Zeitraum eine
Diät durch:

\vspace{2mm}\begin{center}{\tiny\begin{tabular}{l|cccccccccc}
Person Nr.&1&2&3&4&5&6&7&8&9&10\\\hline
Gewicht vorher&
103.7&95.2&87.6&97.7&76.8&110.9&85.1&95.3&101.3&99.8\\
Gewicht nachher&
99.8&92.3&88.2&96.5&74.0&106.4&83.4&95.0&100.0&95.3\\
\end{tabular}}\end{center}\vspace{2mm}

Die Erfinderin der Diät behauptet, dass diese tatsächlich eine Gewichtsabnahme
bewirke.  Prüfen Sie diese Behauptung mit einem statistischen Test nach, 
\begin{outline}
\item mit dem Signifikanzniveau 0.05, 
\item mit dem Signifikanzniveau 0.0005. 
\end{outline}

Geben Sie dabei die Null- und die Alternativhypothese explizit an.
\end{beamerboxesrounded}
\end{frame}

\begin{frame}{Two Sample T-Test, Theorie}
\begin{enumerate}
\item Stichproben: $x_1\ldots x_{n1}$ und $y_1\ldots y_{n2}$.
\item Berechne: $\bar{x}$, $\bar{y}$, $s_1$ und $s_2$ wie üblich.
\item Berechne: 
$$
\nu=n_1+n_2-2,\;
s=\sqrt{\frac{(n_1-1)s_1^2+(n_2-1)s_2^2}{\nu}}
$$

\item Berechne:
$$
t_\text{obs}=\frac{\bar{x}-\bar{y}}{s\sqrt{1/n_1+1/n_2}}
$$
\item Inferenz mit Tabelle oder p-Wert. ($t_\text{obs}\sim T_\nu$).
\end{enumerate}
\end{frame}

\begin{frame}{Two Sample T-Test, Beispiel}
\begin{beamerboxesrounded}[shadow]{Aufgabe 125, Serie 11}
Zwei Gruppen von Ratten erhielten stark bzw. schwach proteinhaltiges Futter.
Die Gewichtszunahmen in Gramm (vom 28. bis 84. Tag) waren bei der ersten Gruppe
\begin{center}
170, 112, 99, 120, 133, 152, 160, 146, 105,
\end{center}
und bei der zweiten Gruppe
\begin{center}
99, 102, 91, 125, 86, 137, 79.
\end{center}

Prüfen Sie die Hypothese H0, dass die beiden Grundgesamtheiten gleiche
Erwartungswerte haben, mit $\alpha = 0.05$.
\end{beamerboxesrounded}
\end{frame}

\section[KI]{Konfidenzintervalle}
\begin{frame}{Übersicht}
\begin{itemize}
\item Im Menü \keystroke{F7}
\item \texttt{2:TInterval\ldots} für den Einstichproben-Fall und
\item \texttt{4:2SampleTInt\ldots} für den Zweistichproben-Fall.
\item Analog zu T-Tests
\item \texttt{Stats} oder \texttt{Data}
\end{itemize}
\end{frame}

\begin{frame}{KI mit \texttt{Stats}}
\begin{beamerboxesrounded}[shadow]{Aufgabe 110, Serie 10}
Eine Stichprobe von 101 Geburten ergab ein mittleres Geburtsgewicht von 3.38 kg
mit einer empirischen Standardabweichung von 0.45 kg. Geben Sie das
Konfidenzintervall für das mittlere Geburtsgewicht an mit Q = 90\%.
\end{beamerboxesrounded}
\end{frame}

\begin{frame}{KI mit \texttt{Data}}
\begin{beamerboxesrounded}[shadow]{Aufgabe 111, Serie 10}
An einer Stichprobe von 12 Spatzen einer bestimmten Art wurden folgende
Flügelspannweiten (in mm) gemessen (siehe Aufgabe 105):

\begin{center}
245, 240, 236, 243, 247, 238, 239, 248, 238, 240, 244, 237
\end{center}

Bestimmen Sie das Konfidenzintervall für die mittlere Flügelspannweite
\begin{outline}
\item mit $Q = 90\%$
\item mit $Q = 80\%$.
\end{outline}
\end{beamerboxesrounded}
\end{frame}

\section{ANOVA}
\begin{frame}{Übersicht}
\begin{itemize}
\item Im Menü \keystroke{F6}: \texttt{C:ANOVA}
\item Eingabe: Eine Spalte pro Gruppe
\item Ausgabe: ANOVA-Tabelle sowie $F-$ und $p-$Wert
\end{itemize}
\end{frame}

\begin{frame}{Theorie}
\begin{enumerate}
\item Gegeben sind $k$ Gruppen, in der $i$-ten Gruppe $n_i$, Beobachtungen
$y_{i\,1}, y_{i\,2}, \ldots, y_{i\,ni}$.
\item Berechne:
$$
n=\sum_{i=1}^k n_i,\;
\bar{Y}_{i.}=\frac{1}{n_i}\sum_{j=1}^{n_i}Y_{ij},\; 
\bar{Y}_{..}=\frac{1}{n}\sum_{i=1}^k\sum_{j=1}^{n_i}Y_{ij}
$$
\item Berechne:
$$
MS_G=\frac{1}{k-1}\underbrace{\sum_{i=1}^k n_i(\bar{Y}_{i.}-\bar{Y}_{..})^2}_{SS_G},\;
MS_E=\frac{1}{n-k}\underbrace{\sum_{i=1}^k\sum_{j=1}^{n_i}(Y_{ij}-\bar{Y}_{i.})^2}_{SS_E}
$$
\item $V=\frac{MS_G}{MS_E}$
\item Inferenz mit Tabelle oder p-Wert. ($V\sim F_{\underbrace{k-1}_{df_G},\underbrace{n-k}_{df_E}}$).
\end{enumerate}
\end{frame}


\begin{frame}{Einfache Varianzanalyse}
\begin{beamerboxesrounded}[shadow]{Aufgabe 130, Serie 12}
Wir messen das Gewicht von Moorfröschen (\textit{Rana arvalis}) an drei verschiedenen
Weiern. Pro Weier vermessen wir vier Moorfrösche und notieren die erhaltenen
Werte (in g) in die folgende Tabelle:

\vspace{2mm}\begin{center}{\scriptsize\begin{tabular}{ccc}\toprule
Weier 1 & Weier 2 & Weier 3\\\midrule
15.3& 17.2& 13.2\\
16.1& 15.8& 15.1\\
15.8& 16.4& 14.9\\
16.3& 16.9& 16.2\\
\bottomrule
\end{tabular}}\end{center}\vspace{2mm}

Unterscheiden sich die Moorfrösche hinsichlich ihres Gewichtes je nachdem an
welchem Weier sie leben? Führen Sie zur Beantwortung dieser Frage eine Varianzanalyse
durch. Wählen Sie als Signifikanzniveau $\alpha = 5\%$.
\end{beamerboxesrounded}
\end{frame}

\section{Regression (II)}

\begin{frame}{Theorie (Inferenz)}
\begin{enumerate}
\item Berechne:
$$
\hat{\sigma}^2 = \frac{1}{n-2}\sum_{i=1}^n (Y_i-\hat{Y}_i)^2
$$
\item Berechne für $\mathcal{H}_0: \beta=b$:
$$
t_\text{obs}=\frac{\hat\beta-b}{\sqrt{\hat\sigma^2/\sum_{i=1}^n(x_i-\bar{x})^2}}
$$
\item Inferenz mit Tabelle oder p-Wert. ($t_\text{obs}\sim T_{n-2}$).
\end{enumerate}
\end{frame}

\begin{frame}{Einfache Lineare Regression}
\begin{beamerboxesrounded}[shadow]{Aufgabe 134, Serie 12}
Wir betrachten die folgenden Messwerte zu fünf der Galápagos-Inseln aus Aufgabe
133:

\vspace{2mm}\begin{center}{\scriptsize\begin{tabular}{l|ccccc}\toprule
Insel & Bartolomé & Darwin & Seymour & Tortuga & Wolf\\\midrule
$x_i$&
1.24&2.33&1.84&1.24&2.85\\
$y_i$&
31&10&44&16&21\\
\bottomrule
\end{tabular}}\end{center}\vspace{2mm}

\begin{itemize}
\item[b)] Bestimmen Sie die Gleichung der Regressionsgeraden
$y=\hat{\alpha}+\hat{\beta}x$ durch diese Punktpaare [\dots].
\item[d)] Hat die Fläche Einfluss auf die Zielgrösse $Y$?
\end{itemize}
Testen Sie H0 : $\beta = 0$ gegen H1 : $\beta \ne 0$
zum Signifikanzniveau 5\%.
\end{beamerboxesrounded}
\end{frame}

%\section{$\chi^2$-Test}
%\begin{frame}{$\chi^2$-Test auf gegebene Verteilung}
%\begin{beamerboxesrounded}[shadow]{Aufgabe 127, Serie 12}
%In einem grossen Wald mit gleichartigem Gelände und Baumbestand wird das
%Vorkommen der Heidelbeere untersucht. Auf quadratischen Flächen von je 1 m$^2$
%Grösse erhält man durch Auszählung der Pflanzen:
%
%\vspace{2mm}\begin{center}{\scriptsize\begin{tabular}{l|cccccccc} 
%Anzahl der Planzen pro m$^2$&0&1&2&3&4&5&6&$\ge 7$\\\hline
%Anzahl der Flächenstücke&15&18&11&4&1&0&1&0
%\end{tabular}}\end{center}\vspace{2mm}
%
%Darf eine Poissonverteilung mit Erwartungswert 1.2 angenommen werden?
%\end{beamerboxesrounded}
%\end{frame}

%\begin{frame}{Übersicht (Test auf Unabhängigkeit)}
%\begin{itemize}
%\item Im Menü \keystroke{F6} mit \texttt{8:Chi2 2-way}
%\item Kontingenztafel als Matrix eingeben:
%\begin{itemize}
%\item Zeileneinträge mit Kommata (\texttt{,}) trennen.
%\item Zeilen mit eckigen Klammern umfassen.
%\item Alles durch ein weiteres Paar eckige Klammern umfassen.
%\end{itemize}
%Beispiel:
%
%\begin{center}
%$\begin{bmatrix}1 & 2\\ 3 & 4\\\end{bmatrix}\Leftrightarrow$\texttt{[[1,2][3,4]]} 
%\end{center}
%
%\item Matrix der erwarteten Werte kann abgespeichert werden.
%\end{itemize}
%\end{frame}
%
%\begin{frame}{Übersicht (Test auf Unabhängigkeit)}
%\begin{enumerate}
%\item Kontingenztafel der Häufigkeiten $N_{ij}$ aufstellen $\Rightarrow$ $m$ Zeilen und $r$ Spalten.
%\item Summe $N$ aller Häufigkeiten berechnen.
%\item Randhäufigkeiten $N_{i.}$ und $N_{j.}$ berechnen.
%\item Erwartete Werte mit
%$$
%E_{ij}=\frac{n_{.i}n_{j.}}{n}
%$$
%berechnen.
%\item
%$$
%u=\sum_{i,j}\frac{(N_{ij}-E_{ij})^2}{E_{ij}}
%$$
%\item Inferenz mit Tabelle oder p-Wert. ($u\sim \chi^2_{(m-1)(r-1)}$).
%\end{enumerate}
%\end{frame}
%
%\begin{frame}{$\chi^2$-Test auf Unabhängigkeit}
%\begin{beamerboxesrounded}[shadow]{Aufgabe 127, Serie 12}
%Am 27. Januar 1987 berichtete die New York Times auf der Titelseite von den Resultaten
%einer Studie uber die präventive Wirkung von Aspirin gegen Herzinfarkte bei Männern
%mittleren Alters. Für die Studie wurden 22071 Männer mittleren Alters zufällig je einer
%von zwei Gruppen zugeordnet. Der einen Gruppe wurde Aspirin verabreicht, der anderen
%ein Placebo. Von 11037 Personen, die Aspirin eingenommen hatten, bekamen 104 einen
%Herzinfarkt; von den 11034 Personen, welchen ein Placebo verabreicht wurde, erlitten 189
%einen Herzinfarkt. Besteht ein signifikanter Unterschied ($\alpha = 0.001$)?
%\end{beamerboxesrounded}
%\end{frame}

\appendix
\section*{MAT191}
\begin{frame}{Inhaltsverzeichnis MAT191}
\begin{enumerate}
\item Die einfache Varianzanalyse (ANOVA);
\item Die einfache lineare Regression, mehrfache Regression, polynomiale Regression;
\item Die zweifache ANOVA (additiv und mit einer Wechselwirkung);
\item Die hierarchische ANOVA (feste und zufällige Effekte);
\item Die Covarianzanalyse mittels einer multiplen Regression;
\item Die nichtparametrischen Verfahren;
\item Relative Häufigkeiten als Daten;
\item Einführung in Maximum Likelihood.
\end{enumerate}
Ausserdem wird eine Einführung in R gegeben.
\end{frame}
\end{document}
