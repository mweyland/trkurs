\documentclass{beamer}
%\documentclass[handout]{beamer}
%\usepackage{pnup}
%\pgfpagesuselayout{3 on 1 with notes}[a4paper,border shrink=3mm]
\mode<presentation>
{
    %\usetheme{Darmstadt}
    \usetheme{Frankfurt}
    %\usetheme{Singapore}
    \usecolortheme{crane}
    \definecolor{craneblue}{RGB}{60,0,0}
%    \usecolortheme{seahorse}
    \setbeamercovered{transparent}
    \setbeamertemplate{items}[triangle]
    \setbeamertemplate{enumerate items}[default]

}


\usepackage[german]{babel}
\usepackage[utf8x]{inputenc}
\usepackage{booktabs}
\setcounter{tocdepth}{2}
\newlength{\tikey}
\newcommand{\keystroke}[1]{\settowidth{\tikey}{\scriptsize #1}\psframebox[framearc=0.2]{\parbox{\tikey}{\scriptsize #1}}}

\usepackage{outline}%
\def\labeloutlni{\theoutlni}%
\def\theoutlni{\protect{(\alph{outlni})}}%
%
%\usepackage{tikz}
%\usebackgroundtemplate{%
%\begin{tikzpicture}
%  \node [rotate=30,scale=8.25,color=gray!20] at (current page.center) {DRAFT};
%\end{tikzpicture}}
\title{Taschenrechnerkurs für MAT183, Teil 1}

\author{Mathias Weyland}

\date{7. Juni -- 13. Juni 2012}

\AtBeginSection[]
{
\subsection{}
  \begin{frame}<beamer>
    \begin{beamerboxesrounded}[shadow]{\inserttitle}
      \insertsection
    \end{beamerboxesrounded}
  \end{frame}
}

\setbeamertemplate{navigation symbols}{}
\begin{document}

\frame{\maketitle}

%\begin{frame}{Punkteverteilung}
%\fig{noten3.eps}
%\end{frame}

\section{Fakultät/Binomialkoeffizient}

\begin{frame}{Aufgabe 3, Serie 1}
An einem Treffen an der Universität Zürich nehmen 18 Professoren der Biologie teil, 5 von
ihnen sind Deutsche, 6 sind Schweizer und 7 sind Amerikaner. Sie möchten 3 Personen
für ein Komitee auswählen.

\begin{outline}
\item Nehmen Sie an, dass es keine Hierarchie innerhalb des Komitees gibt, d.h., dass die
drei Posten nicht voneinander unterscheidbar sind. Wieviele solcher Komitees kann
man wählen? [\ldots]\pause
\item Nehmen Sie nun an, dass es eine Hierarchie innerhalb des Komitees gibt, d.h., dass es
einen Präsidenten, einen Vizepräsidenten und einen Sekretären gibt. Wieviele solcher
Komitees kann man wählen? [\ldots]
\end{outline}
\end{frame}

\section{Summen}
\begin{frame}{Aufgabe 29, Serie 5}
Die Wahrscheinlichkeit dafür, dass ein Schweizer die Blutgruppe B hat, beträgt 7\%. Mit
welcher Wahrscheinlichkeit haben von 20 zufällig ausgewählten Schweizern strikt weniger
als ein Achtel die Blutgruppe B?
\end{frame}

\section{Gleichungen}
\begin{frame}{Aufgabe 15, Serie 3}
Ein Würfel ist verfälscht. Die Zwei ist viermal so wahrscheinlich wie die Sechs, die Eins
und die Drei sind beide doppelt so wahrscheinlich wie die Sechs, und die restlichen beiden
Augenzahlen sind halb so wahrscheinlich wie die Sechs.
\begin{outline}
\item Bestimmen Sie die einzelnen Wahrscheinlichkeiten.
\item $[\dots]$.
\end{outline}
\end{frame}

\section{GLS/Wobei-Operator}
\begin{frame}{Aufgabe 66, Serie 11}
\begin{outline}
\item Bei einer Umfrage unter der weissen Bevölkerung der USA wurde ermittelt, ob
die Leute mit der Amtsführung von George W. Bush zufrieden waren oder nicht.
Es wurden 2000 Leute befragt. Davon gaben 890 Personen an, sie seien mit der
Amtsführung von George W. Bush zufrieden. Die restlichen 1110 Personen gaben an,
sie seien unzufrieden. [\dots]

\item In einer weiteren Umfrage sollte ermittelt werden welcher Anteil der nicht-weissen
US-Bevölkerung mit der Amtsführung von George W. Bush zufrieden war. Diesmal
wurden 3800 Personen befragt. Davon gaben 1492 an, sie seien zufrieden, und die
restlichen gaben an, sie seien dies nicht. Gibt es einen signifikanten Unterschied zwischen der nicht-weissen und weissen US-Bevölkerung, was die Zufriedenheit mit der
Amtsführung von Präsident Bush betrifft ($\alpha = 5\%$).
\end{outline}
\end{frame}

\section{Statistische Masse}
\begin{frame}{Aufgabe 8, Serie 2}
Die Messung der Körperlänge (auf 0.5cm genau) von 16 zehnjährigen Mädchen ergab
folgende Werte (in cm):

\begin{center}
141.5, 143, 140.5, 145.5, 142.5, 143, 151, 146.5, 142, 147.5, 143.5, 144, 142, 145, 146, 144.5
\end{center}
\begin{outline}
\item $[\dots]$
\item $[\dots]$
\item Lagemasse: Bestimmen Sie das arithmetische Mittel, den Median, sowie das 25\%-
und das 75\%-Quartil.
\item Streuungmasse: Bestimmen Sie die Variationsbreite und den Interdezilbereich. [\dots]
\end{outline}
\end{frame}

\section{Integrale}
\begin{frame}{Aufgabe 31, Serie 6}
Die Exponentialverteilung mit Rate $\lambda > 0$ ist durch die Dichtefunktion $f : \mathbb{R} \rightarrow \mathbb{R}$
$$
f(x)=\begin{cases} 0 & \text{für } x < 0\\
     \lambda e^{-\lambda x} & \text{sonst}\\
     \end{cases}
$$
gegeben. 
\begin{outline}
\item $[\dots]$
\item Sei $X$ eine Zufallsgrösse mit Dichtefunktion $f$ . Zeigen Sie, dass die Verteilungsfunktion von $X$ durch
$$
F(x)=\begin{cases} 0 & \text{für } x < 0\\
     1-e^{-\lambda x} & \text{sonst}\\
     \end{cases}
$$
gegeben ist. [\dots]\pause
\item Was ist der Median von X?\pause
\item Bestimmen Sie, für $\lambda = 0.5$, die folgenden Wahrscheinlichkeiten: $ [X\le 2]$, $P[X >
2]$, $P[X\ge 2]$.
\end{outline}
\end{frame}

\begin{frame}{Aufgabe 33, Serie 6}
Die Exponentialverteilung wird oft verwendet, wenn die Zufallsvariable eine “Wartezeit”
misst, also die Zeit $t$ bis zum Eintreffen eines bestimmten Ereignisses. Das klassische
Beispiel dafür ist der Atomzerfall, wo die Dauer bis ein bestimmtes radioaktives Isotop
zerfällt gemessen wird. In einem radioaktiven Gegenstand gibt es viele aktive Isotope.
\vfill
Der Median den wir in Aufgabe 31(c) berechnen, entspricht dem Zeitpunkt $\tilde{t}$, wo ein
einzelnes dieser Isotope mit Wahrscheinlichkeit 50\% zerfallen ist. Im Durchnitt wird also
zu diesem Zeitpunkt $\tilde{t}$ etwa die Hälfte aller Isoptope im radioaktiven Gegenstand zerfallen
und somit die Strahlung auf die Hälfte abgesunken sein. In diesem Zusammenhang ist
der Median einer exponentialverteilten Zufallsgrösse besser bekannt unter dem Begriff
“Halbwertszeit”.
\end{frame}

\begin{frame}{Aufgabe 33, Serie 6 (Fortzsetzung)}
\begin{outline}
\item Radium Ra-226 hat eine Halbwertszeit von 1602 Jahren. Wir stellen die Wartezeit
eines einzelnen Radium-Atoms bis zu seinem Zerfall mit einer exponentialverteilten
Zufallsgrösse $W$ dar. Bestimmen Sie die Rate dieser Verteilung und geben Sie die
Verteilungsfunktion an.
\item[(b)] Mit welcher Wahrscheinlichkeit braucht ein Ra-226 Atom mehr als 4000 Jahre, bis
es zerfallen ist?
\item[(c)] Wir betrachten nun zehn Ra-226 Atome deren Zerfallszeiten $W_1, \dots, W_{10}$ alle voneinander 
unabhängig sind. Mit welcher Wahrscheinlichkeit dauert es weniger als
6000 Jahre, bis alle Atome zerfallen sind?
\end{outline}
\end{frame}
\end{document}
