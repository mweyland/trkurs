\documentclass{beamer}
%\documentclass[handout]{beamer}
%\usepackage{pnup}
%\pgfpagesuselayout{3 on 1 with notes}[a4paper,border shrink=3mm]
\mode<presentation>
{
    %\usetheme{Darmstadt}
    \usetheme{Frankfurt}
    %\usetheme{Singapore}
    \usecolortheme{crane}
    \definecolor{craneblue}{RGB}{60,0,0}
%    \usecolortheme{seahorse}
    \setbeamercovered{transparent}
    \setbeamertemplate{items}[triangle]
    \setbeamertemplate{enumerate items}[default]

}


\usepackage[german]{babel}
\usepackage[utf8x]{inputenc}
\usepackage{booktabs}
\setcounter{tocdepth}{2}
\newlength{\tikey}
\newcommand{\keystroke}[1]{\settowidth{\tikey}{\scriptsize #1}\psframebox[framearc=0.2]{\parbox{\tikey}{\scriptsize #1}}}

\usepackage{outline}%
\def\labeloutlni{\theoutlni}%
\def\theoutlni{\protect{(\alph{outlni})}}%
%
%\usepackage{tikz}
%\usebackgroundtemplate{%
%\begin{tikzpicture}
%  \node [rotate=30,scale=8.25,color=gray!20] at (current page.center) {DRAFT};
%\end{tikzpicture}}
\title{Taschenrechnerkurs für MAT183, Teil 1}

\author{Ueli Hartmann\\Mathias Weyland}

\date{FS13}

\AtBeginSection[]
{
\subsection{}
  \begin{frame}<beamer>
    \begin{beamerboxesrounded}[shadow]{\inserttitle}
      \insertsection
    \end{beamerboxesrounded}
  \end{frame}
}

\setbeamertemplate{navigation symbols}{}
\begin{document}

\frame{\maketitle}

%\begin{frame}{Punkteverteilung}
%\fig{noten3.eps}
%\end{frame}

\section{Fakultät/Binomialkoeffizient}

\begin{frame}{Aufgabe 3, Serie 1}
An einem Treffen an der Universität Zürich nehmen 23 Professoren der Biologie teil, 6 von
ihnen sind Deutsche, 5 sind Schweizer, 5 sind Franzosen und 7 sind Amerikaner.
Sie möchten 4 Personen für ein Komitee auswählen.

\begin{outline}
\item Nehmen Sie an, dass es keine Hierarchie innerhalb des Komitees gibt, d.h., dass die
vier Posten nicht voneinander unterscheidbar sind. Wieviele solcher Komitees kann
man wählen? [\ldots]\pause
\item Nehmen Sie nun an, dass es eine Hierarchie innerhalb des Komitees gibt, d.h., dass es
einen Präsidenten, einen Vizepräsidenten, einen ersten und einen zweiten
Sekretären gibt. Wieviele solcher Komitees kann man wählen? [\ldots]
\end{outline}
\end{frame}

\section{Wobei-Operator}
\begin{frame}{Aufgabe 6, Serie 5}
Für eine Prüfung müssen fünfzehn Stoffgebiete vorbereitet werden, von denen dann
vier zufällig ausgewählt an die Reihe kommen. Eine betroffene Person hat nur
acht dieser Gebiete gelernt. Die Zufallsgrösse $G$ (für gelernt) gibt an,
wieviele der vier Fragen aus einem gelernten Gebiet stammen. Geben Sie die
Verteilung von $G$ in Tabellenform an.
\end{frame}

\section{Gleichungen}
\begin{frame}{Aufgabe 4, Serie 3}
Wie oft darf man eine unverfäschlte Münze höchstens werfen, wenn die Wahrscheinlichkeit
dafür, dass dabei nie Kopf erscheint,
\begin{outline}
\item \dots mindestens 20\% betragen soll?
\item \dots mindestens $10^{-8}$ betragen soll?
\end{outline}
\end{frame}

\begin{frame}{Aufgabe 3, Serie 3}
Ein Glücksrad soll aus einem roten, einem blauen und einem grünen Sektor
bestehen. Dabei soll die Wahrscheinlichkeit für ``rot oder grün'' gleich jener
für ``blau'' und die Wahrscheinlichkeit für ``grün oder blau'' doppelt so gross
wie jene für ``rot'' sein.
\begin{outline}
\item Bestimmen Sie die einzelnen Wahrscheinlichkeiten $[\dots]$.
\item $[\dots]$.
\end{outline}
\end{frame}

\section{Statistische Masse}
\begin{frame}{Aufgabe 3, Serie 2}
Die Messung der Körperlänge (auf 0.5cm genau) von 16 zehnjährigen Mädchen ergab
folgende Werte (in cm):

\begin{center}
131.5, 130, 137.5, 140.5, 132.5, 151, 138, 136, 133, 138, 134.5, 140, 132, 139, 133.5, 131
\end{center}
\begin{outline}
\item $[\dots]$
\item $[\dots]$
\item Lagemasse: Bestimmen Sie das arithmetische Mittel, den Median, sowie das 25\%-
und das 75\%-Quartil.
\item Streuungmasse: Bestimmen Sie die Variationsbreite und den Interdezilbereich. [\dots]
\item Streuungmasse: Bestimmen Sie die empirische Varianz und die Standardabweichung [\dots].
\end{outline}
\end{frame}

\section{Integrale}
\begin{frame}{Aufgabe 2, Serie 6}
Gegeben sei folgende Funktion $f : \mathbb{R }\rightarrow \mathbb{R}$,
$$
f(x):=\begin{cases} ax-0.5 & \text{für } 1 \le x \le 3\\
     0 & \text{sonst}\\
     \end{cases}
$$
gegeben. 
\begin{outline}
\item Für welchen Wert der Konstanten $a$ ist $f(x)$ die Dichtefunktion einer Zufallsgrösse $X$?
\item Geben Sie die Verteilungsfunktion $F$ von $X$ an.
\item Bestimmen Sie den Median von X.
\end{outline}
\end{frame}
\end{document}
